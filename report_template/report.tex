\documentclass[11pt,dvipdfmx]{jarticle}

\usepackage{eee}
\usepackage{subfig}
\usepackage{graphicx}
\usepackage{pdfpages}
\usepackage{float}
\usepackage{amsmath}
\usepackage{amssymb}
\usepackage{color}
\usepackage{multirow}
\usepackage{adjustbox}
\usepackage{url}
\usepackage{mymacros}
\usepackage{here}

\begin{document}

%表紙%
%%%%%%%%%%%%%%%%%%%%%%%%%%%%%%%%%%%%%%%%%%%%%
\includepdf[noautoscale=true]{Title.pdf}
%%%%%%%%%%%%%%%%%%%%%%%%%%%%%↑↑ここのファイル名を表紙PDFファイル名に変更

\section{目的}
本実験の目的は、時間信号と周波数スペクトルの相互関係を理解するとともに、アナログフィルタを用いた周波数選択によって時間信号が変化する様子を理解することである。

\section{原理}

\subsection{フーリエ級数}
周期 $T$ の周期関数 $f(t)$ は、以下のような三角関数の級数で表される:
\begin{align}
f(t) = a_0 + \sum_{n=1}^{\infty} a_n \cos \frac{2\pi nt}{T} + \sum_{n=1}^{\infty} b_n \sin \frac{2\pi nt}{T}
\end{align}

各係数は以下のように求められる:
\begin{align}
a_0 &= \frac{1}{T} \int_{-T/2}^{T/2} f(t) dt \\
a_n &= \frac{2}{T} \int_{-T/2}^{T/2} f(t) \cos \frac{2\pi nt}{T} dt \\
b_n &= \frac{2}{T} \int_{-T/2}^{T/2} f(t) \sin \frac{2\pi nt}{T} dt
\end{align}

\subsection{伝達関数とRCフィルタ}
線形システムの動作は伝達関数によって記述され、周波数 $f_n$ に対して次のように定義される:
\begin{align}
H(f_n) = |H(f_n)| e^{j\theta(f_n)}
\end{align}

RC低域通過フィルタ(LPF)の伝達関数は、
\begin{align}
H(f) = \frac{1}{1 + j\frac{f}{f_1}}
\end{align}
ただし、$f_1 = \frac{1}{2\pi RC}$ は遮断周波数である。

RC高域通過フィルタ(HPF)の伝達関数は、
\begin{align}
H(f) = \frac{1}{1 + j\frac{f_2}{f}}
\end{align}
ただし、$f_2 = \frac{1}{2\pi RC}$ である。

\subsection{離散フーリエ変換(DFT)}
時間領域の離散信号 $f_n$ に対して、離散フーリエ変換は以下の式で与えられる:
\begin{align}
F_k = \sum_{n=0}^{N-1} f_n e^{-j \frac{2\pi kn}{N}}
\end{align}

\section{実験方法}

\subsection{時間信号の測定}
VirtualBenchを用いて、以下の手順で時間信号を測定した:

\begin{enumerate}
  \item \wfig{実験回路1}に示すとおりに結線した。
  \item DUTとして\wtab{各種LCフィルタ}に示すLPFを接続した。
  \item FGENで正弦波を選択し、$10\,\mathrm{kHz}$から$1\,\mathrm{MHz}$までの周波数、および遮断周波数における入力・出力波形を測定し、数値データとして保存した。
  \item 各班員が用意した任意波形、および班で用意したガウシアンパルスをFGENに設定し、それぞれ測定を行い、数値データとして保存した。
  \item DUTを\wtab{各種LCフィルタ}に示すHPF, BPF, BEFへ変更し、同様の手順で測定を繰り返した。
  \item 測定結果から、各周波数における振幅比(dB)および位相差を計算し、グラフを作成した。
  \item 任意波形およびガウシアンパルスについては、入力信号と出力信号を同一グラフにプロットし、波形の変化を確認できるようにした。
\end{enumerate}
\begin{figure}[H]
  \centering
  \includegraphics[width=0.8\textwidth]{fig/VirtualBench.drawio.png}
  \caption{時間信号測定用の実験回路}
  \label{fig:実験回路1}  
\end{figure}
\begin{table}[H]
  \centering
  \caption{各種LCフィルタ}
  \label{tab:各種LCフィルタ}
  \begin{tabular}{|c|m{0.22\textwidth}|c|}
    \hline
    フィルタの種類 & 回路図 & 素子値の計算 $f_c$ \\
    \hline
    LPF & \includegraphics[width=0.2\textwidth]{fig/LPF.drawio.png} & $f_c = 150\,\mathrm{kHz}$ \\
    \hline
    HPF & \includegraphics[width=0.2\textwidth]{fig/HPF.drawio.png} & $f_c = 140\,\mathrm{kHz}$ \\
    \hline
    BPF & \includegraphics[width=0.2\textwidth]{fig/BPF.drawio.png} & $f_{c1} = 50\,$kHz,$f_{c2} = 200\,$kHz \\
    \hline
    BEF & \includegraphics[width=0.2\textwidth]{fig/BEF.drawio.png} & $f_{c1} = 100\,$kHz,$f_{c2} = 140\,$kHz \\
    \hline
  \end{tabular}
\end{table}
\subsection{周波数特性の測定}
NanoVNAを用いてフィルタの周波数特性を測定した。以下の手順で実施した:

\begin{enumerate}
  \item \wfig{実験回路2}に示すとおりに結線した。
  \item ポートをCOM3に設定し、測定周波数範囲を$50\,\mathrm{kHz}$から$1\,\mathrm{MHz}$に設定した。
  \item キャリブレーション(open, short, load, isolation, through)を行い、「save0」に記録した。
  \item DUTとして各フィルタ(LPF, HPF, BPF, BEF)を接続し、測定データを「s2p」形式で保存した。
\end{enumerate}
\begin{figure}[H]
  \centering
  \includegraphics[width=0.8\textwidth]{fig/VNA.drawio.png}
  \caption{周波数特性測定用の実験回路}
  \label{fig:実験回路2}
\end{figure}

\section{実験結果}
\subsection{時間信号の特性}
\subsubsection{LPF}
LPFの入力・出力波形を以下に示す。LPFでは、入力信号と出力信号の波形がどのように変化するかを確認した。

% LPF関連グラフ
\begin{figure}[H]
  \centering
  \includegraphics[width=0.8\textwidth]{fig/10kHz_LPF_characteristic.png}
  \caption{10kHz LPF波形}
  \label{fig:10kHz_LPF}
\end{figure}
\begin{figure}[H]
  \centering
  \includegraphics[width=0.8\textwidth]{fig/20kHz_LPF_characteristic.png}
  \caption{20kHz LPF波形}
  \label{fig:20kHz_LPF}
\end{figure}
\begin{figure}[H]
  \centering
  \includegraphics[width=0.8\textwidth]{fig/40kHz_LPF_characteristic.png}
  \caption{40kHz LPF波形}
  \label{fig:40kHz_LPF}
\end{figure}
\begin{figure}[H]
  \centering
  \includegraphics[width=0.8\textwidth]{fig/50kHz_LPF_characteristic.png}
  \caption{50kHz LPF波形}
  \label{fig:50kHz_LPF}
\end{figure}
\begin{figure}[H]
  \centering
  \includegraphics[width=0.8\textwidth]{fig/80kHz_LPF_characteristic.png}
  \caption{80kHz LPF波形}
  \label{fig:80kHz_LPF}
\end{figure}
\begin{figure}[H]
  \centering
  \includegraphics[width=0.8\textwidth]{fig/100kHz_LPF_characteristic.png}
  \caption{100kHz LPF波形}
  \label{fig:100kHz_LPF}
\end{figure}
\begin{figure}[H]
  \centering
  \includegraphics[width=0.8\textwidth]{fig/150kHz_LPF_characteristic.png}
  \caption{150kHz LPF波形}
  \label{fig:150kHz_LPF}
\end{figure}
\begin{figure}[H]
  \centering
  \includegraphics[width=0.8\textwidth]{fig/200kHz_LPF_characteristic.png}
  \caption{200kHz LPF波形}
  \label{fig:200kHz_LPF}
\end{figure}
\begin{figure}[H]
  \centering
  \includegraphics[width=0.8\textwidth]{fig/400kHz_LPF_characteristic.png}
  \caption{400kHz LPF波形}
  \label{fig:400kHz_LPF}
\end{figure}
\begin{figure}[H]
  \centering
  \includegraphics[width=0.8\textwidth]{fig/1000kHz_LPF_characteristic.png}
  \caption{1000kHz LPF波形}
  \label{fig:1000kHz_LPF}
\end{figure}
\wfig{10kHz_LPF}から\wfig{1000kHz_LPF}より、100kHzまでは入力と出力の振幅の性殆どないが、150kHzを超えると出力側の振幅が減少しているのがわかる。

\begin{figure}[H]
  \centering
  \includegraphics[width=0.8\textwidth]{fig/Gaussian_pulse_LPF_characteristic.png}
  \caption{Gaussian pulse LPF波形}
  \label{fig:Gaussian_pulse_LPF}
\end{figure}
\begin{figure}[H]
  \centering
  \includegraphics[width=0.8\textwidth]{fig/Original_wave1_LPF_characteristic.png}
  \caption{Original wave1 LPF波形}
  \label{fig:Original_wave1_LPF}
\end{figure}
\begin{figure}[H]
  \centering
  \includegraphics[width=0.8\textwidth]{fig/Original_wave2_LPF_characteristic.png}
  \caption{Original wave2 LPF波形}
  \label{fig:Original_wave2_LPF}
\end{figure}
\begin{figure}[H]
  \centering
  \includegraphics[width=0.8\textwidth]{fig/Original_wave3_LPF_characteristic.png}
  \caption{Original wave3 LPF波形}
  \label{fig:Original_wave3_LPF}
\end{figure}
\begin{figure}[H]
  \centering
  \includegraphics[width=0.8\textwidth]{fig/Original_wave4_LPF_characteristic.png}
  \caption{Original wave4 LPF波形}
  \label{fig:Original_wave4_LPF}
\wfig{Gaussian_pulse_LPF}から\wfig{Original_wave4_LPF}までの波形を比較すると、LPFによって高周波成分が除去され、波形が平滑化されていることがわかる。
\end{figure}
\subsubsection{HPF}
HPFの入力・出力波形を以下に示す。HPFでは、入力信号と出力信号の波形がどのように変化するかを確認した。

% HPF関連グラフ
\begin{figure}[H]
  \centering
  \includegraphics[width=0.8\textwidth]{fig/10kHz_HPF_characteristic.png}
  \caption{10kHz HPF波形}
  \label{fig:10kHz_HPF}
\end{figure}
\begin{figure}[H]
  \centering
  \includegraphics[width=0.8\textwidth]{fig/20kHz_HPF_characteristic.png}
  \caption{20kHz HPF波形}
  \label{fig:20kHz_HPF}
\end{figure}
\begin{figure}[H]
  \centering
  \includegraphics[width=0.8\textwidth]{fig/40kHz_HPF_characteristic.png}
  \caption{40kHz HPF波形}
  \label{fig:40kHz_HPF}
\end{figure}
\begin{figure}[H]
  \centering
  \includegraphics[width=0.8\textwidth]{fig/50kHz_HPF_characteristic.png}
  \caption{50kHz HPF波形}
  \label{fig:50kHz_HPF}
\end{figure}
\begin{figure}[H]
  \centering
  \includegraphics[width=0.8\textwidth]{fig/80kHz_HPF_characteristic.png}
  \caption{80kHz HPF波形}
  \label{fig:80kHz_HPF}
\end{figure}
\begin{figure}[H]
  \centering
  \includegraphics[width=0.8\textwidth]{fig/100kHz_HPF_characteristic.png}
  \caption{100kHz HPF波形}
  \label{fig:100kHz_HPF}
\end{figure}
\begin{figure}[H]
  \centering
  \includegraphics[width=0.8\textwidth]{fig/140kHz_HPF_characteristic.png}
  \caption{140kHz HPF波形}
  \label{fig:150kHz_HPF}
\end{figure}
\begin{figure}[H]
  \centering
  \includegraphics[width=0.8\textwidth]{fig/200kHz_HPF_characteristic.png}
  \caption{200kHz HPF波形}
  \label{fig:200kHz_HPF}
\end{figure}
\begin{figure}[H]
  \centering
  \includegraphics[width=0.8\textwidth]{fig/400kHz_HPF_characteristic.png}
  \caption{400kHz HPF波形}
  \label{fig:400kHz_HPF}
\end{figure}
\begin{figure}[H]
  \centering
  \includegraphics[width=0.8\textwidth]{fig/1000kHz_HPF_characteristic.png}
  \caption{1000kHz HPF波形}
  \label{fig:1000kHz_HPF}
\end{figure}
\wfig{10kHz_HPF}から\wfig{1000kHz_HPF}より、低周波成分が除去されていることがわかる。

\begin{figure}[H]
  \centering
  \includegraphics[width=0.8\textwidth]{fig/Gaussian_pulse_HPF_characteristic.png}
  \caption{Gaussian pulse HPF波形}
  \label{fig:Gaussian_pulse_HPF}
\end{figure}
\begin{figure}[H]
  \centering
  \includegraphics[width=0.8\textwidth]{fig/Original_wave1_HPF_characteristic.png}
  \caption{Original wave1 HPF波形}
  \label{fig:Original_wave1_HPF}
\end{figure}
\begin{figure}[H]
  \centering
  \includegraphics[width=0.8\textwidth]{fig/Original_wave2_HPF_characteristic.png}
  \caption{Original wave2 HPF波形}
  \label{fig:Original_wave2_HPF}
\end{figure}
\begin{figure}[H]
  \centering
  \includegraphics[width=0.8\textwidth]{fig/Original_wave3_HPF_characteristic.png}
  \caption{Original wave3 HPF波形}
  \label{fig:Original_wave3_HPF}
\end{figure}
\begin{figure}[H]
  \centering
  \includegraphics[width=0.8\textwidth]{fig/Original_wave4_HPF_characteristic.png}
  \caption{Original wave4 HPF波形}
  \label{fig:Original_wave4_HPF}
\end{figure}
\wfig{Gaussian_pulse_HPF}から\wfig{Original_wave4_HPF}までの波形を比較すると、HPFによって低周波成分が除去され、波形が鋭くなっていることがわかる。
\subsubsection{BPF}
% BPF関連グラフ
\begin{figure}[H]
  \centering
  \includegraphics[width=0.8\textwidth]{fig/10kHz_BPF_characteristic.png}
  \caption{10kHz BPF波形}
  \label{fig:10kHz_BPF}
\end{figure}
\begin{figure}[H]
  \centering
  \includegraphics[width=0.8\textwidth]{fig/20kHz_BPF_characteristic.png}
  \caption{20kHz BPF波形}
  \label{fig:20kHz_BPF}
\end{figure}
\begin{figure}[H]
  \centering
  \includegraphics[width=0.8\textwidth]{fig/40kHz_BPF_characteristic.png}
  \caption{40kHz BPF波形}
  \label{fig:40kHz_BPF}
\end{figure}
\begin{figure}[H]
  \centering
  \includegraphics[width=0.8\textwidth]{fig/50kHz_BPF_characteristic.png}
  \caption{50kHz BPF波形}
  \label{fig:50kHz_BPF}
\end{figure}
\begin{figure}[H]
  \centering
  \includegraphics[width=0.8\textwidth]{fig/80kHz_BPF_characteristic.png}
  \caption{80kHz BPF波形}
  \label{fig:80kHz_BPF}
\end{figure}
\begin{figure}[H]
  \centering
  \includegraphics[width=0.8\textwidth]{fig/100kHz_BPF_characteristic.png}
  \caption{100kHz BPF波形}
  \label{fig:100kHz_BPF}
\end{figure}
\begin{figure}[H]
  \centering
  \includegraphics[width=0.8\textwidth]{fig/200kHz_BPF_characteristic.png}
  \caption{200kHz BPF波形}
  \label{fig:200kHz_BPF}
\end{figure}
\begin{figure}[H]
  \centering
  \includegraphics[width=0.8\textwidth]{fig/400kHz_BPF_characteristic.png}
  \caption{400kHz BPF波形}
  \label{fig:400kHz_BPF}
\end{figure}
\begin{figure}[H]
  \centering
  \includegraphics[width=0.8\textwidth]{fig/1000kHz_BPF_characteristic.png}
  \caption{1000kHz BPF波形}
  \label{fig:1000kHz_BPF}
\end{figure}
\wfig{10kHz_BPF}から\wfig{1000kHz_BPF}より、通過帯域以外の成分が除去されていることがわかる。

\begin{figure}[H]
  \centering
  \includegraphics[width=0.8\textwidth]{fig/Gaussian_pulse_BPF_characteristic.png}
  \caption{Gaussian pulse BPF波形}
  \label{fig:Gaussian_pulse_BPF}
\end{figure}
\begin{figure}[H]
  \centering
  \includegraphics[width=0.8\textwidth]{fig/Original_wave1_BPF_characteristic.png}
  \caption{Original wave1 BPF波形}
  \label{fig:Original_wave1_BPF}
\end{figure}
\begin{figure}[H]
  \centering
  \includegraphics[width=0.8\textwidth]{fig/Original_wave2_BPF_characteristic.png}
  \caption{Original wave2 BPF波形}
  \label{fig:Original_wave2_BPF}
\end{figure}
\begin{figure}[H]
  \centering
  \includegraphics[width=0.8\textwidth]{fig/Original_wave3_BPF_characteristic.png}
  \caption{Original wave3 BPF波形}
  \label{fig:Original_wave3_BPF}
\end{figure}
\begin{figure}[H]
  \centering
  \includegraphics[width=0.8\textwidth]{fig/Original_wave4_BPF_characteristic.png}
  \caption{Original wave4 BPF波形}
  \label{fig:Original_wave4_BPF}
\end{figure}
\wfig{Gaussian_pulse_BPF}から\wfig{Original_wave4_BPF}までの波形を比較すると、BPFによって特定帯域以外の成分が除去されていることがわかる。
\subsubsection{BEF}
BEFの入力・出力波形を以下に示す。BEFでは、入力信号と出力信号の波形がどのように変化するかを確認した。
% BEF関連グラフ
\begin{figure}[H]
  \centering
  \includegraphics[width=0.8\textwidth]{fig/10kHz_BEF_characteristic.png}
  \caption{10kHz BEF波形}
  \label{fig:10kHz_BEF}
\end{figure}
\begin{figure}[H]
  \centering
  \includegraphics[width=0.8\textwidth]{fig/20kHz_BEF_characteristic.png}
  \caption{20kHz BEF波形}
  \label{fig:20kHz_BEF}
\end{figure}
\begin{figure}[H]
  \centering
  \includegraphics[width=0.8\textwidth]{fig/40kHz_BEF_characteristic.png}
  \caption{40kHz BEF波形}
  \label{fig:40kHz_BEF}
\end{figure}
\begin{figure}[H]
  \centering
  \includegraphics[width=0.8\textwidth]{fig/50kHz_BEF_characteristic.png}
  \caption{50kHz BEF波形}
  \label{fig:50kHz_BEF}
\end{figure}
\begin{figure}[H]
  \centering
  \includegraphics[width=0.8\textwidth]{fig/80kHz_BEF_characteristic.png}
  \caption{80kHz BEF波形}
  \label{fig:80kHz_BEF}
\end{figure}
\begin{figure}[H]
  \centering
  \includegraphics[width=0.8\textwidth]{fig/100kHz_BEF_characteristic.png}
  \caption{100kHz BEF波形}
  \label{fig:100kHz_BEF}
\end{figure}
\begin{figure}[H]
  \centering
  \includegraphics[width=0.8\textwidth]{fig/140kHz_BEF_characteristic.png}
  \caption{140kHz BEF波形}
  \label{fig:140kHz_BEF}
\end{figure}
\begin{figure}[H]
  \centering
  \includegraphics[width=0.8\textwidth]{fig/200kHz_BEF_characteristic.png}
  \caption{200kHz BEF波形}
  \label{fig:200kHz_BEF}
\end{figure}
\begin{figure}[H]
  \centering
  \includegraphics[width=0.8\textwidth]{fig/400kHz_BEF_characteristic.png}
  \caption{400kHz BEF波形}
  \label{fig:400kHz_BEF}
\end{figure}
\begin{figure}[H]
  \centering
  \includegraphics[width=0.8\textwidth]{fig/1000kHz_BEF_characteristic.png}
  \caption{1000kHz BEF波形}
  \label{fig:1000kHz_BEF}
\end{figure}
\wfig{10kHz_BEF}から\wfig{1000kHz_BEF}より、特定帯域が除去されていることがわかる。

\begin{figure}[H]
  \centering
  \includegraphics[width=0.8\textwidth]{fig/Gaussian_pulse_BEF_characteristic.png}
  \caption{Gaussian pulse BEF波形}
  \label{fig:Gaussian_pulse_BEF}
\end{figure}
\begin{figure}[H]
  \centering
  \includegraphics[width=0.8\textwidth]{fig/Original_wave1_BEF_characteristic.png}
  \caption{Original wave1 BEF波形}
  \label{fig:Original_wave1_BEF}
\end{figure}
\begin{figure}[H]
  \centering
  \includegraphics[width=0.8\textwidth]{fig/Original_wave2_BEF_characteristic.png}
  \caption{Original wave2 BEF波形}
  \label{fig:Original_wave2_BEF}
\end{figure}
\begin{figure}[H]
  \centering
  \includegraphics[width=0.8\textwidth]{fig/Original_wave3_BEF_characteristic.png}
  \caption{Original wave3 BEF波形}
  \label{fig:Original_wave3_BEF}
\end{figure}
\begin{figure}[H]
  \centering
  \includegraphics[width=0.8\textwidth]{fig/Original_wave4_BEF_characteristic.png}
  \caption{Original wave4 BEF波形}
  \label{fig:Original_wave4_BEF}
\end{figure}
\wfig{Gaussian_pulse_BEF}から\wfig{Original_wave4_BEF}までの波形を比較すると、BEFによって特定帯域の成分が除去されていることがわかる。
\subsection{周波数特性の測定}
NanoVNAを用いて測定した各フィルタの周波数特性を以下に示す。
\begin{figure}[H]
  \centering
  \includegraphics[width=0.8\textwidth]{fig/NanoVNA_LPF.png}
  \caption{LPFの周波数特性(NanoVNA測定)}
  \label{fig:NanoVNA_LPF}
\end{figure}
\wfig{NanoVNA_LPF}より、LPFの遮断周波数が約150kHzであること、LPFとして機能していることが確認できる。
\begin{figure}[H]
  \centering
  \includegraphics[width=0.8\textwidth]{fig/NanoVNA_HPF.png}
  \caption{HPFの周波数特性(NanoVNA測定)}
  \label{fig:NanoVNA_HPF}
\end{figure}
\wfig{NanoVNA_HPF}より、HPFの遮断周波数が約140kHzであること、HPFとして機能していることが確認できる。
\begin{figure}[H]
  \centering
  \includegraphics[width=0.8\textwidth]{fig/NanoVNA_BPF.png}
  \caption{BPFの周波数特性(NanoVNA測定)}
  \label{fig:NanoVNA_BPF}
\end{figure}
\wfig{NanoVNA_BPF}より、BPFの通過帯域が約50kHzから200kHzであること、BPFとして機能していることが確認できる。
\begin{figure}[H]
  \centering
  \includegraphics[width=0.8\textwidth]{fig/NanoVNA_BEF.png}
  \caption{BEFの周波数特性(NanoVNA測定)}
  \label{fig:NanoVNA_BEF}
\end{figure}
\wfig{NanoVNA_BEF}より、BEFの通過帯域が約100kHzから140kHzであること、BEFとして機能していることが確認できる。
\section{考察}
\subsection{フィルタ(ノイズ除去)の実例を示せ}

\subsubsection{horlabs社の製品のBNC型ローパスフィルターとBNC型ハイパスフィルター}
この2つの製品はBNCコネクタを持つパッシブフィルターであり、主にオシロスコープに接続して使用される。

\subsubsection{MJCETの研究であるBEFフィルタを用いた60 Hzの電源ノイズ除去}

この研究では、BEFフィルタを用いて60 Hzの電源ノイズを除去する方法が提案されている。特に医療現場での心電図(ECG)信号のノイズ除去に焦点を当てており、
高品質、低サイズ、低消費電力での設計がされている。

\subsection{時間信号の考察}
\subsubsection{LPF}
\wfig{10kHz_LPF}から\wfig{1000kHz_LPF}より、周波数が150\ kHzより高くなると遮断されている。
このことから、遮断周波数より高い周波数成分を持つ信号を遮断することが出来ると考えられる。
\subsubsection{HPF}
\wfig{10kHz_HPF}から\wfig{1000kHz_HPF}より、周波数が140\ kHzより低くなると遮断されている。
このことから、遮断周波数より低い周波数成分を持つ信号を遮断することが出来ると考えられる。
\subsubsection{BPF}
\wfig{10kHz_BPF}から\wfig{1000kHz_BPF}より、周波数が50\ kHzより低くなると遮断されている。
また、周波数が200\ kHzより高くなると遮断されている。
このことから、BPFは特定の周波数帯域を通過させることが出来ると考えられる。
\subsubsection{BEF}
\wfig{10kHz_BEF}から\wfig{1000kHz_BEF}より、周波数が100\ kHzより低くなると遮断されている。
また、周波数が140\ kHzより高くなると遮断されている。
このことから、BEFは特定の周波数帯域を除去することが出来ると考えられる。
\subsection{周波数特性の考察}

\subsection{時間信号と周波数特性の比較}

\subsection{任意波形の考察}

\subsection{ガウシアンパルスを用いる理由}
本実験でガウシアンパルスを用いる理由として、その生成の容易さ、数学的な扱いやすさ、そして理想的な形状であることが挙げられる。
ガウシアンパルスは比較的シンプルな回路や機器で生成可能であり、実験系の構築を容易にする。
また、その形状はガウス関数で表され、微分や積分が容易であるため、データ解析や理論モデルとの比較をシンプルに行える。
さらに、パルスがなめらかでオーバーシュートやリンギングが発生しないため、信号の歪みを最小限に抑えたい精密な実験に適している。
以上の理由から、本実験ではガウシアンパルスを用いて時間信号の測定を行ったと考えられる
\subsection{Sパラメータの調査}

\subsection{動画とフーリエ級数展開の関係}
動画\footnote{\url{https://www.nicovideo.jp/watch/sm32671229}}とフーリエ級数展開の関係を説明するにあたって、まず離散フーリエ変換と図の描画について説明する。
描画したい図を点の集合とし、それぞれの点を複素平面的に考えx+jyの形にする。
それらの点は連続しているためx座標に関する式とy座標に関する式の2つが立式できる。
これらの式をフーリエ係数に変換するのが離散フーリエ変換である。
また、図の再構成はこうして得られたフーリエ係数を逆離散フーリエ変換することで可能である。
そうすることでフーリエ係数で定義された正弦波と余弦波を合成した一つの式を導ける。
これらの
\subsection{独自考察}

\section{結論}
か

\section{謝辞}
き


\begin{thebibliography}{99}
\bibitem{1} Maliha Naaz,Mohammed Arifuddin Sohel,Kaleem Fatima,M.A. Raheem:低消費電力ノッチフィルタの生体医療応用への設計,International Journal of Innovative Research in Electrical, Electronics, Instrumentation and Control Engineering, Vol.4, No.12, pp.75–79 (2016). DOI: 10.17148/IJIREEICE.2016.41215
\end{thebibliography}

\end{document}